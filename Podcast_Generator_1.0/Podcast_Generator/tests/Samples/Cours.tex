\documentclass[a4paper]{article}
\usepackage{lscape}
\usepackage{graphicx,wrapfig}
\usepackage[T1]{fontenc}
\usepackage[utf8]{inputenc}
\usepackage[a4paper,left=0.8cm,right=0.8cm,top=1.5cm,bottom=1cm]{geometry}
\usepackage{float}
\usepackage{multicol}
\usepackage{fancyvrb} %Verbatims Encadrés
\usepackage{tabularx}
\usepackage[version=4]{mhchem}
\pagenumbering{gobble}

\renewcommand{\figurename}{Doc.}
\setlength{\columnsep}{0.1cm}


\begin{document}
\begin{LARGE}
\begin{center}
Enseignement Scientifique - Réforme 2023\\
Sciences Physiques\\[0.2cm]
\hrule
\vspace*{0.2cm}
Chapitre 1
\end{center}
\end{LARGE}

\begin{Large}
\noindent Un \underline{être vivant} est composé \underline{d'organes}, qui sont eux-mêmes faits de \underline{tissus}. Les tissus sont un assemblage de \underline{cellules}, qui sont formées par des \underline{molécules}. Enfin, les molécules sont simplement des \underline{atomes} mis ensemble. Dans les atomes, il y a des composants plus petits.
\section*{A] Structure de l'Atome}
Un atome (taille $\sim 10^{-10}$ m) est constitué d'un noyau ($\sim 10^{-15}$ m) et d'électrons présents autour du noyau. Ca veut dire que le noyau est $10\,000$ fois plus petit que l'atome.\\
\begin{minipage}{0.35\textwidth}
\includegraphics[scale=0.5]{atome.png}
\end{minipage}
\begin{minipage}{0.65\textwidth}
\begin{itemize}
\item Noyau : Composé de \underline{$Z$ \textbf{protons} ($+$)} et de \underline{$(A-Z)$ \textbf{neutrons}}. Leurs taille et masse similaires nous invitent à les classer dans la même catégorie : les nucléons (au nombre de $A$).\\[1cm]
\item Autour : \underline{$Z$ \textbf{électrons} ($-$)} organisés en couches électroniques ($(1s), (2s), (2p), (3s), ...$)
\end{itemize} 
\end{minipage}

\vspace*{1cm}
Remarques :
\begin{itemize}
\item Un atome est électriquement neutre : il y a \underline{autant de protons ($+$) que d'électrons ($-$)}.
\item C'est le \underline{nombre de protons $Z$ qui fait la nature d'un atome}. 26 protons fait qu'un atome s'appelle Fer ; un atome de 11 protons sera obligatoirement du Sodium.
\item \underline{On symbolise un atome par $\ce{^A_ZX}$}. Par exemple $\ce{^{27}_{13}Al}$ symbolise l'Aluminium, qui a $Z=13$ protons, et $A=27$ nucléons (donc $A-Z=14$ neutrons).
\end{itemize}
\vfill
Exercice :
\begin{enumerate}
\item Si un atome a 17 nucléons et 10 neutrons, combien a-t-il de protons ? et d'électrons ?
\item Comment symboliser un atome d'Oxygène qui a 8 protons, 9 neutrons et 8 électrons ?
\end{enumerate}

\newpage
\section*{B] Abondances}
Il apparaît pertinent de vouloir visualiser la proportion d'atomes (sur Terre, dans un être vivant, dans l'Univers, ..). On peut tout autant utiliser une courbe, un diagramme en bâton, ou circulaire.
\begin{figure}[H]
\begin{center}
\includegraphics[scale=0.7]{abun1.png}
\end{center}
\end{figure}
\begin{figure}[H]
\begin{center}
\includegraphics[scale=1]{abun2.png}
\end{center}
\end{figure}

\noindent On se rend compte que ce n'est pas toujours les mêmes atomes qu'on va retrouver en grande quantité : Dans l'Univers, c'est surtout de l'Hydrogène ($H$) et de l'Helium ($He$) (: les carburants des étoiles). \\
Chez les humains, c'est surtout $O$ (67\%), $C$ (18\%), $H$ (10\%), et $N$ (3\%).

\section*{C] Transformations}
En Physique-Chimie, on rencontre trois types de transformations qui seront symbolisées de manière similaire :
\begin{enumerate}
\item \underline{Réactions Chimiques} : exemple : {\huge $\ce{2 H_2 + O_2 -> 2 H_2O}$}.\\
Ici on retrouve les mêmes atomes (les mêmes lettres) à gauche et à droite de l'équation-bilan. Il s'agit donc d'un réarrangement d'atomes déjà existants.\\
\item \underline{Réactions Physiques} = Changement d'Etat : exemple : {\huge $\ce{H_2O_{(l)} -> H_2O_{(g)}}$}.\\
Ici on ne change pas d'atome/molécule/ion. On change d'état de la matière (solide / liquide / gaz) en fonction des liaisons électromagnétiques faibles (voir Ch2).\\
\item \underline{\textbf{Réactions Nucléaires}} : exemple : {\huge $\ce{^{228}_{90}Th -> ^{224}_{88}Ra + ^4_2\alpha}$}.\\
Dans ce type de réaction, la nature des atomes change. Ici un atome de Thorium devient un atome de Radon.\\
\end{enumerate}

\noindent Dans les prochains chapitres (Ch1 et Ch4), notre objet d'étude sera la troisième catégorie : les réactions nucléaires.\\ 
Dans cette catégorie, il y a \underline{3 réactions nucléaires} :
\begin{enumerate}
\item \underline{La fusion nucléaire} (cf Ch4)
\item \underline{La fission nucléaire}
\item \underline{La radioactivité (=la désintégration radioactive)} (cf Ch1)\\
\end{enumerate}

\vfill

\noindent Exercice :\\
Dans les trois réactions suivantes, est-ce qu'il s'agit de transformations chimiques, physiques ou nucléaires ?
\begin{enumerate}
\item \qquad {\LARGE $\ce{Be_{(g)} + He_{(g)} -> C_{(g)}}$}.
\item \qquad {\LARGE $\ce{CH_4_{(g)} + 2O_2_{(g)} -> 2H_2O_{(g)} + CO2_{(g)}}$}.
\item \qquad {\LARGE $\ce{N_2_{(g)} -> N_{2(l)}}$}.
\end{enumerate}


\newpage
\section*{D] Fusion et Fission}
Tous les atomes sont différents, certains atomes sont plus stables que d'autres. On observe :
\begin{figure}[H]
\vspace*{-0.4cm}
\begin{center}
\includegraphics[scale=0.6]{fusfiss.png} 
\end{center}
\end{figure}
\vspace*{-0.6cm}
On remarque que \underline{l'élement le plus stable est le Fer} ($\ce{^{56}_{26}Fe}$). Dès lors:
\begin{itemize}
\item Tous les atomes \underline{à gauche du Fer} (plus légers, dont le nombre de protons < 26) gagnent à se rapprocher du Fer. Ils vont chercher à grossir : c'est la \textbf{fusion nucléaire} (étoiles/ITER).
\item Tous les atomes \underline{à droite du Fer} (plus lourds, dont le nombre de protons > 26) gagnent à se rapprocher du Fer. Ils vont chercher à rapetisser : c'est la \textbf{fission nucléaire} (centrales nucléaires).
\end{itemize}

\begin{figure}[H]
\vspace*{-1cm}
\begin{center}
\includegraphics[scale=0.7]{fusfiss2.png} 
\end{center}
\end{figure}
\vspace*{-0.7cm}
\noindent En se rapprochant du Fer, ces atomes libérent de l'énergie autour d'eux (dans les centrales et les étoiles, les atomes donnent de la lumière et de la chaleur).

\section*{E] Radioactivité}
Si seul $Z$ est responsable de la nature d'un atome, on peut imaginer plusieurs versions d'un même atome. Par exemple :
\begin{itemize}
\item 6 protons / 6 neutrons donnera un noyau de Carbone 12 ($\ce{^{12}_6C}$)
\item 6 protons / 8 neutrons donnera un noyau de Carbone 14 ($\ce{^{14}_6C}$)
\item 6 protons / 12 neutrons donnera un noyau de Carbone 18 ($\ce{^{18}_6C}$)
\end{itemize}
On appelle ces \underline{différentes versions d'un même atome : des isotopes}.\\

\noindent Même chose concernant l'Helium : 
\begin{figure}[H]
\vspace*{-0.4cm}
\begin{center}
\includegraphics[scale=1]{helium.png} 
\end{center}
\end{figure}
\vspace*{-1.4cm}
\noindent Dans les trois cas, il y a toujours 2 protons, c'est donc de l'Helium, mais on remarque que les trois isotopes sont différents.\\

\noindent \underline{Certains des isotopes existants sont instables : ils vont se désintégrer}.\\

\noindent Il existe trois types de désintégration radioactive :
\begin{enumerate}
\item Désintégration $\beta +$ (beta plus) : émission d'un positron $e^+$
\item Désintégration $\beta -$ (beta moins) : émission d'un électron $e^-$
\item Désintégration $\alpha$ (alpha) : émission d'une particule alpha (=noyau d'Hélium)\\
\end{enumerate}

\noindent Le phénomène de désintégration est aléatoire et imprévisible (cf Lancer de dés), mais en prenant un grand nombre d'atomes on peut visualiser une loi statistique.\\
Plus le temps passe et moins il reste d'atomes à désintégrer, et donc de moins en moins d'atomes se désintègrent.\\

\noindent On obtient alors ce genre de courbe de désintégration :
\begin{figure}[H]
\begin{center}
\includegraphics[scale=0.5]{iode.png}
\end{center}
\end{figure}
\noindent Au lieu de réfléchir en terme de probabilité de désintégration (possible mais pas évident), on caractérise une courbe par une propriété graphique : \underline{le temps de demi-vie (noté $t_{1/2}$)}.\\

\noindent \underline{Le temps de demi-vie ($t_{1/2}$) c'est le temps} (en années, jours, secondes...) \underline{aubout duquel la} \underline{moitié d'un stock d'atome s'est désintégré}.\\
Au bout de $t_{1/2}$ (ans, jours, secondes), on a perdu la moitié du stock initial.\\

\noindent La courbe (exponentielle décroissante) fait en sorte qu'à chaque fois qu'on attend $t_{1/2}$ (ans, jours, secondes) on reperd la moitié. Ainsi, on peut facilement faire un tableau d'évolution :\\
\begin{center} 
\begin{tabular}{l | c | c | c | c | c | c | c }
Pourcentage d'atomes & 100\% & 50\% & 25\% & 12.5\% & 6.25\% & 3.13\% & 1.56\% \\
\hline
Temps (en demi-vies) & 0 & $1\times t_{1/2}$ & $2\times t_{1/2}$ & $3\times t_{1/2}$ & $4\times t_{1/2}$ & $5\times t_{1/2}$ & $6\times t_{1/2}$ \\
\end{tabular}
\end{center}
Entre chaque colonne du tableau, on divise par deux la proportion d'atomes restants, et il s'écoule $t_{1/2}$.\\
\underline{Il en résulte qu'il restera toujours des atomes ne s'étant pas encore désintégrés}.\\

\vfill
\noindent Exemple :\\
Le Césium 137 ($\ce{^{137}Cs}$) met environ $30$ ans pour que la moitié de son stock se désintègre :

\begin{center} 
\begin{tabular}{l | c | c | c | c | c | c | c }
Pourcentage d'atomes & 100\% & 50\% & 25\% & 12.5\% & 6.25\% & 3.13\% & 1.56\% \\
\hline
Temps (en demi-vies) & 0 & 30 ans & 60 ans & 90 ans & 120 ans & 150 ans & 180 ans \\
\end{tabular}
\end{center}


\newpage
\section*{Application] Datation au Carbone 14}
Le Carbone 14 ($\ce{^{14}_{6}C}$) est un isotope du Carbone qui est radioactif : il se désintègre donc. \\
Il est absorbé par tous les êtres vivants (animaux/végétaux) via la nourriture et la respiration.\\
La perte et l'ajoût de $\ce{^{14}C}$ s'équilibrent, la proportion de Carbone 14 dans l'organisme est alors constante.\\
A la mort de l'être vivant, il cesse de respirer et de s'alimenter. La proportion ne fait alors que décroître.
\begin{figure}[H]
\begin{center}
\includegraphics[scale=0.73]{c14.JPG}
\end{center} 
\end{figure}
\noindent En $\sim 5730$ ans, on passe de $100\%$ à $50\%$ de Carbone 14. On écrit donc $t_{1/2}=5730$ ans.\\
Si on attend de nouveau $5730$ ans, on redivise par deux : il reste $25\%$ de Carbone 14.\\

\noindent Si on mesure la proportion de Carbone 14 dans un échantillon, on peut estimer la date de la mort de l'organisme.\\
La courbe nous montre que le Carbone 14 n'est pas idéal pour mesurer des morts récentes (entre 0 et quelques centaines d'années), ou des morts très lointaines (après quelques dizaines de milliers d'années).\\

\noindent On évite alors d'utiliser le Carbone 14 pour dater des morts récentes (crimes par exemple) ou bien des morts très anciennes (dinosaures par exemple).

\newpage
\section*{Application] Imagerie médicale}
Les désintégrations radioactives peuvent provoquer des pathologies pour les humains (lésions cutanées, brûlures, défaillances d'organes, cancer, mort).\\
Néanmoins, il existe plusieurs mécanismes de radioactivité, et certains types sont jugés moins dangereux que d'autres.\\

\noindent On se sert de certaines transformations nucléaires dans le domaine médical, à des fins diagnostiques. C'est le cas des examens de :
\begin{itemize}
\item \underline{scintigraphie} avec du Technicium 99 ($^{99m}Tc$ : $t_{1/2} = 6\,h$).
\item \underline{TEP} (ou \emph{PET Scan} en anglais) pour Tomographie par Emission de Positons, basée sur du Fluor 18 ($^{18}F : t_{1/2} = 2\,h$).
\end{itemize}

\noindent Les principes sont relativement similaires : On injecte dans le corps du patient une dose d'un radioisotope, puis ces atomes seront absorbés pour l'élaboration de cellules. \\A l'aide d'une caméra gamma, on peut visualiser à quels endroits ces nouveaux atomes se sont amassés : c'est une zone qui fabrique beaucoup de cellules, c'est-à-dire une potentielle tumeur.\\

\begin{figure}[H]
\begin{center}
\includegraphics[scale=0.45]{scinti.png}
\end{center} 
\end{figure}

\noindent Comme les isotopes se désintègrent très vite, on diminue les risques de dégâts à long terme chez le patient. En contrepartie, les hôpitaux préparent une dose pour une heure donnée, et tout retard du patient devient problématique en terme de logistique.

\end{Large}
\end{document}


